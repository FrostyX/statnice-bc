\documentclass[10pt,a4paper]{article}
\usepackage[utf8]{inputenc}
\usepackage[czech]{babel}
\usepackage[T1]{fontenc}
<<<<<<< HEAD
\usepackage{fullpage}
=======

\title{Teoretické základy informatiky}
>>>>>>> 6b9067b7c8b951d889c2ad66d92fa9c7a1d23eb8
\begin{document}
\maketitle



\newpage

\section{První odstavec}

Formální jazyky a jejich hierarchie. Regulární jazyky (definice, uzávěrové vlastnosti). Konečné automaty deterministické
a nedeterministické. Regulární výrazy, automaty s epsilon-přechody. Minimalizace konečného deterministického
automatu. Pumping lemma. Bezkontextové jazyky a jejich vlastnosti (uzávěrové vlastnosti, jednoznačnost).
Zásobníkové automaty a jejich modifikace. Deterministické zásobníkové automaty. Deterministické
bezkontextové jazyky.

%============================================================================
%                                                                             DRUHÝ ODSTAVEC
%============================================================================

\section{Druhý odstavec}

Turingův stroj (TS), nedeterministický TS. Jazyk přijímaný TS, jazyk rozhodovaný TS. Church-Turingova
teze, varianty TS. Částečně rekurzivní a rekurzivní jazyky, jazyky a rozhodovací problémy. Vztah rekurzivních
a částečně rekurzivních jazyků. Uzávěrové vlastnosti jazyků TS. Riceova věta. Vztah jazyků TS k jazykům
Chomského hierarchie. Věta o rekurzi.

	\subsection{Turingův stroj (TS)}

		
		\begin{itemize}
			\item Alan Turing, 1936
			\item účel: porozumět omezením mechanického výpočtu
		\end{itemize}		
		\textbf{TS se skládá z}
		\begin{itemize}
			\item řídící jednotky, která se vždy nachází v jednom z konečného množství stavů
			\item zleva omezené nekonečné pásky rozdělené na políčka, v každém políčku je zapsán jeden symbol
			\item čtecí / zapisovací hlavy která je vždy umístěna nad jedním políčkem pásky
		\end{itemize}
		\textbf{Definice: }
		Turingův stroj je struktura $\langle Q, \Sigma, \Gamma, \delta, q_{start}, q_{+}, q_{-} \rangle $ dána:
		\begin{itemize}
			\item Neprázdnou konečnou množinou stavů $Q$
			\item Vstupní abecedou  $\Sigma$ t.ž. \textvisiblespace $ \notin \Sigma$
			\item páskovou abecedou $\Gamma$ t.ž. $\Sigma \subset \Gamma,$ \textvisiblespace $\in  \Gamma $
			\item přechodovou funkcí $\delta : Q \times  \Gamma \rightarrow Q \times \Gamma \times \{L , R \} $
			\item počátečním stavem $q_{start} \in Q$
			\item přijímacím stavem $q_{+} \in Q $ a zamítacím stavem $q_{-} \in Q.$ $q_{+} \neq q_{-}$
		\end{itemize}
		Program TS lze chápat jako množinu elementárních instrukcí ve tvaru: 

		\textit{\uv{Pokud je řídící jednotka ve stavu $q$ a čtecí/zapisovací hlava čte symbol $a$, tak změň stav řídící jednotky
		na $q'$, na pásku zapiš $a'$ a posuň
		čtecí/zapisovací hlavu o jedno políčko směrem $d$.}}

		Takováto instrukce se zapisuje jako $\delta (q,a) = (q', a', d)$ a nazýváme ji přechod. Celý program, tedy množinu 				takovýchto instrukcí, pak nazýváme přechodovou funkcí TS

		\textbf{Konfigurace} TS je uspořádaná trojice $(q, \alpha, n) \in Q \times \Gamma^{*} \times N_{0}$ , která zachycuje
		aktuální status všech tří komponent. 
		\begin{itemize}
			\item $q$ je aktuální stav řídící jednotky
			\item $\alpha$ je obsah pásky
			\item $n$ je pozice hlavy
		\end{itemize}

		\textbf{Krok výpočtu TS} je definován jako binární relace na množině konfigurací: Nechť $(q,a_{0} \dots a_{n}, i) $
		je taková konfigurace $T$, kde $q \neq q_{\pm}, n \in N_{0}, a_{0}, \dots, a_{n}~\in~\Gamma, i~\leq~n. $
		\begin{itemize}
			\item Je-li $1 \leq i \leq n a \delta(q,a_{i}) = (q', b, L), pak$
				$$(q,a_{0} \dots a_{n}, i) \vdash (q', a_{0} \dots a_{i-1}ba_{i+1} \dots a_{n}, i-1) $$
			\item Je-li $\delta(q,a_{0}) = (q', b, L), pak$
				$$(q,a_{0} \dots a_{n}, 0) \vdash (q', ba_{1} \dots a_{n}, 0) $$
			\item Je-li $\delta(q,a_{i}) = (q', b, R), pak$
				$$(q,a_{0} \dots a_{n}, i) \vdash (q', a_{0} \dots a_{i-1}ba_{i+1} \dots a_{n}, i+1) $$
		\end{itemize}
	
	

	\subsection{Nedeterministický TS}

		Obdobný rozdíl jako u konečných deterministických a konečných nedeterministických automatů. Přechodová funkce ve 				tvaru:
		$$\delta : Q \times  \Gamma \rightarrow 2^{Q \times \Gamma \times \{L , R \}}$$

		\textbf{Definice: }
		Nedeterministický TS je struktura $\langle Q, \Sigma, \Gamma, \delta, q_{start}, q_{+}, q_{-} \rangle $ dána:
		\begin{itemize}
			\item Neprázdnou konečnou množinou stavů $Q$
			\item Vstupní abecedou  $\Sigma$ t.ž. \textvisiblespace $ \notin \Sigma$
			\item páskovou abecedou $\Gamma$ t.ž. $\Sigma \subset \Gamma,$ \textvisiblespace $\in  \Gamma $
			\item přechodovou funkcí $\delta : Q \times  \Gamma \rightarrow 2^{Q \times \Gamma \times \{L , R \}} $
			\item počátečním stavem $q_{start} \in Q$
			\item přijímacím stavem $q_{+} \in Q $ a zamítacím stavem $q_{-} \in Q.$ $q_{+} \neq q_{-}$
		\end{itemize}

		Ke každému deterministickému existuje ekvivalentní nedeterministický automat a ke každému 
		nedeterministickému existuje ekvivalentní deterministický automat.

	\subsection{Jazyk přijímaný TS}

		Množinu všech slov $\omega \in \Sigma^{*}$, které TS \textit{T} Přijímá značíme \textit{L(T)} a nazýváme 					\textit{\textbf{jazykem Turingova stroje}}, t.j.
		 $$L(T) = \{ \omega | \omega \in \Sigma^{*}, (\omega, q_{0},0) \vdash^{*} C_{+}\} $$
		Jazyk $L(T)$ nazýváme \textit{jazyk přijímaný TS} $T$.\\
		Říkáme, že TS $T$ \textit{přijímá jazyk}$ L(T)$.\\		 
	
		Jazyk $L \subseteq \Sigma^{*} $ nazveme\textit{ jazyk přijímaný TS}, pokud existuje TS $T$ který jej přijímá.

	\subsection{Jazyk rozhodovaný TS}

		Pokud navíc platí, že TS $T$ zamítá každé $\omega \notin L(T)$, nazýváme jazyk L(T) 
		\textit{jazyk~rozhodovaný} TS $T$.\\
		Říkáme, že TS $T$ \textit{rozhoduje jazyk} $L(T)$. (to znamená že nikdy necyklí)

		Jazyk $L \subseteq \Sigma^{*} $ nazveme\textit{ jazyk rozhodovaný TS}, pokud existuje TS $T$ který jej rozhoduje.
	

	\subsection{Churg-Turingova teze}

		\textit{Intuitivní pojem algoritmu = algoritmus TS.} \\(to je všechno co k tomu máme)

	\subsection{Varianty TS}
		\begin{itemize}
			\item TS, který se nikdy nepokusí přejet levý okraj pásky
				\begin{itemize}
					\item Rozumíme TS, u kterého při výpočtu nad jakým koli slovem nedojde k tomu že, je v konfiguraci 
						$(q,a\omega,0)$ a existuje přechod $\delta(q,a) = (q',b,L).$  Tedy nikdy nenastane situace
						že by byla hlava nad nejlevějším políčkem pásky a přechodová funkce by určovala pohyb 								vlevo.
				\end{itemize}
			\item TS, který nikdy nezapíše na pásku \textvisiblespace
				\begin{itemize}
					\item Roziníme TS, který nemá žádný přechod ve tvaru $\delta(q,a) = (q',\textvisiblespace,D)$
				\end{itemize}
			\item TS, který po sobě \uv{který po sobě uklízí}
				\begin{itemize}
					\item Roziníme TS, který zastaví pouze v konfiguraci $\langle q_{+}, \epsilon, 0 \rangle$
						nebo $\langle q_{-}, \epsilon, 0 \rangle$. Tzn. smaže obsah pásky.
				\end{itemize}
			\item \textbf{TS s instrukcí stop} --- struktura  $\langle Q, \Sigma, \Gamma, \delta, q_{start}, q_{+}, q_{-} \rangle $
								dána:
				\begin{itemize}
					\item Neprázdnou konečnou množinou stavů $Q$
					\item Vstupní abecedou  $\Sigma$ t.ž. \textvisiblespace $ \notin \Sigma$
					\item páskovou abecedou $\Gamma$ t.ž. $\Sigma \subset \Gamma,$ \textvisiblespace $\in  \Gamma $
					\item přechodovou funkcí $\delta : Q \times  \Gamma \rightarrow Q \times \Gamma \times \{L, S, R \} $
					\item počátečním stavem $q_{start} \in Q$
					\item přijímacím stavem $q_{+} \in Q $ a zamítacím stavem $q_{-} \in Q.$ $q_{+} \neq q_{-}$
				\end{itemize}
				Odpovídajícím způsobem se upraví definice kroku výpočtu, vlatně se přidá bod: 

				-- Je-li $\delta(q, a_{i}) = (q', b, S)$, pak 
					$$(q,a_{0} \dots a_{n}, i) \vdash (q', a_{0} \dots a_{i-1}ba_{i+1} \dots a_{n},i) $$
			\item \textbf{TS s oboustranně nekonečnou páskou} --- definice je totožná jako u klasického TS; rozdíl je v definici
			 konfigurace (nevystačíme si s $\langle q, \omega, i \rangle \in Q \times \Gamma^{*} \times N_{0}$)a výpočtu
			(není zarážení o levý okraj)
				\begin{itemize}
					\item \textbf{Konfigurace:} Alternativně zapisujeme jako řetězec
					 	$\alpha q\beta \in \Gamma^{*}Q\Gamma^{*}.$ 
						Konfigurace $\alpha q\beta$ představuje status stroje, který má na pásce zapsán řetězec
						$ \alpha\beta$,
						hlava	je nad prvním symbolem řetězce $\beta$, řídící jednotka je ve stavu q.

						U TS s oboustranně nekonečnou páskou považujeme $\alpha q\beta$ za totožnou s  
						$\alpha q\beta$\textvisiblespace a s \textvisiblespace$\alpha q\beta$.
					\item \textbf{Krok výpočtu TS (s oboustranně nekonečnou páskou)} je definován jako binární 
						relace na množině konfigurací: Nechť $\alpha aqB\beta$ je taková konfigurace $T$, kde 
						$q \neq q, \alpha, \beta \in \Gamma^{*}, a,b \in \Gamma.$
						\begin{itemize}
							\item Je-li $\delta(q,a) = (q', x, L) $, pak  $$\alpha aqb\beta \vdash \alpha q'ax\beta$$.
							\item Je-li $\delta(q,a_{1}) = (q', x, R) $, pak 
									$$\alpha aqb\beta \vdash \alpha axq'\beta$$.
						\end{itemize}
				\end{itemize}
					\item \textbf{TS s více páskami} --- je struktura $\langle Q, \Sigma, \Gamma, \delta, q_{0}, q_{+}, 
						q_{-} \rangle $ dána:
						\begin{itemize}
							\item Neprázdnou konečnou množinou stavů $Q$
							\item Vstupní abecedou  $\Sigma$ t.ž. \textvisiblespace $ \notin \Sigma$
							\item páskovou abecedou $\Gamma$ t.ž. $\Sigma \subset \Gamma,$ \textvisiblespace 
								$\in  \Gamma $
							\item přechodovou funkcí $\delta : Q \times  \Gamma^{k} \rightarrow Q \times [ \Gamma 
								\times \{L, S, R \} ]^{k}$
							\item počátečním stavem $q_{start} \in Q$
							\item přijímacím stavem $q_{+} \in Q $ a zamítacím stavem $q_{-} 
								\in Q.$ $q_{+} \neq q_{-}$
						\end{itemize}		
			
		\end{itemize}
		Ke každé variantě TS vždy existuje ekvilalentní klasický TS a obráceně.

	

	\subsection{Částečně rekurzivní a rekurzivní jazyky}
	
	\subsection{Jazyky a rozhodovací problémy}

	\subsection{Vztah rekurzivních a částečne rekurzivních jazyků}

	\subsection{Uzávěrové vlastnosti jazyků TS}

	\subsection{Riceova věta}

	\subsection{Vztah jazyků TS k jazykům Chomského hierarchie}

	\subsection{Věta o rekurzi}


%============================================================================
%                                                                             TŘETÍ ODSTAVEC
%============================================================================

\section{Třetí odstavec}

Složitost algoritmu (časová a paměťová). Třída P, třída NP, důvody jejich zavedení, jejich vzájemný vztah. NP-
úplné problémy. Cook-Levinova věta. Příklady NP-úplných problémů, dokazovaní NP-úplnosti. Třída PSPACE,
její vztah k třídám P a NP, PSPACE-úplné problémy. Třídy N a NL a NL-úplné problémy.

	\subsection{Složitost algoritmu(časová a paměťová)}

	\subsection{Třída P, NP, důvody jejich zavedení, vzájemný vztah}

	\subsection{NP--úplné problémy}

	\subsection{Cook--Levinova věta}

	\subsection{Příklady NP--úplných problémů}

	\subsection{Dokazování NP--úplnosti}

	\subsection{Třída PSPACE, její vztah k třídám P a NP}

	\subsection{PSPACE--úplné problémy}

	\subsection{Třídy N, NL a NL--úplné problémy}


%============================================================================
%                                                                            ČTVRTÝ ODSTAVEC
%============================================================================

\section{Čtvrtý odstavec}

Výroková logika: jazyk, formule, pravdivostní ohodnocení, tautologie, tabulková metoda, sémantické vyplývání,
normální formy formulí, úplné systémy spojek. Axiomatický systém výrokové logiky, syntaktické vyplývání. Věta
o dedukci. Věty o korektnosti a úplnosti výrokové logiky. Predikátová logika: jazyk, termy a formule, struktury
pro jazyk, ohodnocení termů a formulí. Axiomatický systém predikátové logiky, syntaktické vyplývání. Věty o
korektnosti a úplnosti predikátové logiky. Neklasické logiky, fuzzy logika. Základy logického programování, úvod
do Prologu.


	\subsection{Výroková Logika}


		\subsubsection{ jazyk}
 
		\subsubsection{pravdivostní ohodnocení}
 
		\subsubsection{tautologie}
 
		\subsubsection{tabulková metoda}
 
		\subsubsection{sémantické vyplývání}
 
		\subsubsection{normální formy formulí}
 
		\subsubsection{úplné systémy spojek}

		\subsubsection{axiomatický systém výrokové logiky}

		\subsubsection{syntaktické výplývání}

		\subsubsection{Věta o dedukci}

		\subsubsection{Věty o korektnosti a úplnosti výrokové logiky}

	\subsection{Predikátová logika}

		\subsubsection{jazyk}

		\subsubsection{termy a formule}

		\subsubsection{struktury pro jazyk}

		\subsubsection{ohodnocení termu a formulí}

		\subsubsection{axiomatický systém predikátové logiky}

		\subsubsection{syntaktické vyplývání}

		\subsubsection{věty o korektnosti a úplnosti predikátové logiky}

	\subsection{Neklasické logiky}

		\subsubsection{fuzzy logika}

	\subsection{Základy logického programování, úvod do Prologu}
	
			

\end{document}

\documentclass[10pt,a4paper]{article}
\usepackage[utf8]{inputenc}
\usepackage[czech]{babel}
\usepackage[T1]{fontenc}
\usepackage{epsfig}
\usepackage{listings}
\usepackage{color}
\usepackage{hyperref}
\usepackage{ amssymb }
\usepackage{ mathrsfs }
\usepackage{amsthm}
\usepackage{amsmath}
\usepackage{amsfonts}
\newtheorem{veta}{Věta}
\newtheorem{definition}{Definice}
\newtheorem{note}{Poznámka}




\begin{document}

\title{Státnicový okruh 3: \\ Algoritmizace a programování}
\maketitle
\newpage
\tableofcontents
\newpage

\section{}
\paragraph{Lineární datové struktury: seznam, zásobník, fronta. Úloha třídění a rozdělení třídících algoritmů. Metody třídění porovnáváním: insert sort, Select sort, Bubble sort, Quick sort, Merge sort, Heap sort. Další metody třídění: Counting sort, Radix Sort, Bucket sort, vnější třídění. Složitosti třídicích algoritmů. Pořádkové statistiky.}
\newpage

\section{}
\paragraph{Grafy, stromy, základní pojmy a tvrzení. Vyhledávání a rozdělení vyhledávacích algoritmů. Vyhledávání v lineárních datových strukturách. Binární vyhledávací stromy, průchod a vyhledávání. Red-black stromy, AVL-stromy, B-stromy a jejich struktura, operace vyhledání, vložení a zrušení prvku. Hashování: hashovací funkce, organizace tabulek a způsoby řešení konfliktů.}

\subsection{Grafy}
Grafický způsob vyjádření vztahů mezi nějakými objekty. Objekty jsou v grafu reprezentovány uzly. Vztahy jsou v grafu reprezentovány hranami. Způsob kreslení volíme především tak, aby graf byl přehledný. Hrana vždy začíná a končí v nějakém uzlu. Většinou jsou koncové uzly hrany různé, ale může to být i stejný uzel, pak takové hraně říkáme \textbf{smyčka}. Dva uzly mohou být spojeny více hranami – takovým hranám říkáme \textbf{násobné}. Graf, ve kterém mezi některými uzly je více hran, nazýváme \textbf{multigraf}.
\\ \\
Hrany v grafu mohou být:
\begin{itemize}
	\item \textbf{Neorientované} reprezentují symetrické vztahy mezi uzly. Například v situaci, kdy Petr a Jana jsou bratr a sestra a hrana nám bude v grafu vyjadřovat tento vztah, že jsou sourozenci, tato hrana bude neorientovaná.
	\item \textbf{Orientované} reprezentují jednosměrné, nesymetrické vztahy mezi uzly. Orientace hrany je vyznačena šipkou na jednom jejím konci. Například je-li Eva matkou Petra a Jany a hrany nám budou vyjadřovat vztah, že Eva je jejich rodičem, bude tyto hrany orientované.
\end{itemize}
Podle typu hran dělíme grafy na:
\begin{itemize}
	\item \textbf{Neorientované} – všechny jejich hrany jsou neorientované
	\item \textbf{Orientované} – všechny jejich hrany jsou orientované
	\item \textbf{Smíšené} – obsahují neorientované i orientované hrany
\end{itemize}
\begin{definition}
Graf je dvojice G = (U, H), kde \\
\begin{itemize}
	\item[] U je množina uzlů $U = \{ u_1, u_2, \cdots, u_m \}$\\
	\item[] H je množina hran $H = \{ h_1, h_2, \cdots, h_n \}$
\end{itemize}
\end{definition}
U neorientovaného grafu jsou hrany určeny neuspořádanými dvojicemi svých koncových uzlů $h_i = \{ u_j, u_k\}$, kde $u_j, u_k \in U$ \\
U orientovaného grafu jsou hrany určeny uspořádanými dvojicemi svých koncových uzlů $h_i = \langle u_j, u_k \rangle$, kde $u_j, u_k \in U$ \\
Počet hran se zančí $|H|$. Počet uzlů $|U|$.
\begin{definition}
\textbf{Stupeň uzlu} je počet hran, které jsou s tímto uzel spojeny - mají tento uzel jako koncový. Pro skutečnost, že hrana má daný uzel jako koncový, používáme termín, že hrana s tímto uzlem \textbf{inciduje}.
\end{definition}
U orientovaného grafu navíc rozeznáváme výstupní stupeň uzlu, což je počet hran, které z něho vychází, označujeme ho $d^-(u)$, a vstupní stupeň uzlu, což je počet hran, které do něho vchází, označuje ho $d^+(u)$. Zřejmě pro stupeň uzlu v orientovaném grafu platí $d(u) = d^-(u) + d^+(u) $.
\begin{definition}
V teorii grafů se používají termíny:
\begin{itemize}
	\item Pro uzel, jehož stupeň je nulový, používáme název diskrétní uzel.
	\item Graf, jehož všechny uzly jsou diskrétní, nazveme diskrétní graf.
	\item Dva uzly, jež jsou spojeny hranou, nazveme sousedními uzly.
	\item Pro graf, jehož všechny uzly jsou sousední (má při daném počtu uzlů maximální počet hran), se používá označení úplný graf.
\end{itemize}
\end{definition}
\begin{definition}
Nechť je dán graf $G = (U, H)$. Pak graf $G_1 = (U_1 , H_1 )$ takový, že $U_1 \subset U$ a $H_1 \subset H$, nazveme \textbf{podgrafem grafu G}.
\end{definition}
\paragraph{Reprezentace grafu v programech}
Graf si můžeme v programu reprezentovat různými způsoby. Můžeme si ho například uložit jako matici sousednosti. Ovšem u rozsáhlejších grafů s větším počtem uzlů je tato matice značně velká a navíc její použití v algoritmech je poměrně neefektivní, neboť v ní musíme pracně hledat.
\paragraph{Reprezentace pomocí polí}
Struktura grafu je uložena ve dvou polích. První pole má stejný počet prvků, jako je počet uzlů v grafu. Každému uzlu odpovídá jeden prvek pole. V něm je uložena hodnota indexu, od kterého v druhém poli začíná seznam uzlů, jež jsou sousedé tohoto uzlu.
\paragraph{Reprezentace dynamickou datovou strukturou}
Další možnost je uzel grafu reprezentovat jako strukturovaný datový typ. Každý uzel obsahuje pole (seznam) ukazatelů na sousední uzly.
\paragraph{Izomorfismus grafů}
\begin{definition}
Neorientované grafy $G_1 = (U_1, H_1)$ a $G_2 = (U_2, H_2)$ jsou izomorfní právě když existuje bijektivní zobrazení $h: U_1 \rightarrow U_2$, pro které platí $\{u, v\} \in H_1$ právě když $\{h(u), h(v)\} \in H_2$.
\end{definition}
\begin{definition}
Orientované grafy $G_1 = (U_1, H_1)$ a $G_2 = (U_2, H_2)$ jsou izomorfní právě když existuje bijektivní zobrazení $h: U_1 \rightarrow U_2$, pro které platí $\langle u, v\rangle \in H_1$ právě když $\langle h(u), h(v)\rangle \in H_2$.
\end{definition}
\url{http://upload.wikimedia.org/wikipedia/commons/9/9a/Graph_isomorphism_a.svg} \\
\url{http://upload.wikimedia.org/wikipedia/commons/8/84/Graph_isomorphism_b.svg} \\
Je zřejmé, že izomorfní grafy musí mít stejný počet uzlů, stejný počet hran, stejný počet uzlů daného stupně atd.
\subsection{Stromy}





\end{document}

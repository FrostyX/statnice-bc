\documentclass[10pt,a4paper]{article}
\usepackage[utf8]{inputenc}
\usepackage[czech]{babel}
\usepackage[T1]{fontenc}
\usepackage{epsfig}
\usepackage{listings}
\usepackage{color}
\usepackage{hyperref}
\usepackage{ amssymb }
\usepackage{ mathrsfs }
\usepackage{amsthm}
\usepackage{amsmath}
\usepackage{amsfonts}
\newtheorem{veta}{Věta}
\newtheorem{definition}{Definice}
\newtheorem{note}{Poznámka}




\begin{document}

\title{Státnicový okruh 4: \\ Informační technologie}
\maketitle
\newpage
\tableofcontents
\newpage

%-----------------------------------prvni odstavec------------------------------------
\section{}
\paragraph{John von Neumannova a harvardská architektura počítače, princip jeho činnosti. Binární logika, logické operace a funkce, logické obvody. Reprezentace čísel a znaků v paměti počítače. Osobní počítač (PC), základní deska, chipset a sběrnice (interní, externí). Procesor (CPU), vykonávání instrukcí, podprogramy a zásobník, přerušení. Paměti počítače (RAM, cache, disk, diskové pole). Přídavné karty PC, datové mechaniky a média (CD, DVD, paměťové karty), periferie.}



\newpage
%-----------------------------------druhy odstavec------------------------------------
\section{}
\paragraph{Operační systém, architektura, poskytovaná rozhraní. Správa procesoru: procesy a vlákna, plánování jejich
běhu, komunikace a synchronizace. Problém uváznutí, jeho detekce a metody předcházení. Správa operační pa-
měti: segmentace, stránkování, virtuální paměť. Správa diskového prostoru: oddíly, souborové systémy, zajištění
konzistence dat.}




\newpage
%-----------------------------------treti odstavec------------------------------------
\section{}
\paragraph{Klasifikace (LAN/MAN/WAN) a služby počítačových sítí. Síťová architektura TCP/IP a referenční model
ISO OSI. Strukturovaná kabeláž, přepínaný Ethernet a WLAN. Protokol IP, adresa a maska, směrování, IP
multicast. Protokoly TCP a UDP, správa spojení a řízení toku dat. Systém DNS, překlad jména (na IP adresu,
reverzní), protokol. Služby WWW, elektronické pošty, přenosu souborů a vzdáleného přihlášení.}
\subsection{Klasifikace (LAN/MAN/WAN)}
\end{document}

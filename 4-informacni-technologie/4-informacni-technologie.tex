\documentclass[10pt,a4paper]{article}
\usepackage[utf8]{inputenc}
\usepackage[czech]{babel}
\usepackage[T1]{fontenc}
\usepackage{epsfig}
\usepackage{listings}
\usepackage{color}
\usepackage{hyperref}
\usepackage{ amssymb }
\usepackage{ mathrsfs }
\usepackage{amsthm}
\usepackage{amsmath}
\usepackage{amsfonts}
\newtheorem{veta}{Věta}
\newtheorem{definition}{Definice}
\newtheorem{note}{Poznámka}




\begin{document}

\title{Státnicový okruh 4: \\ Informační technologie}
\maketitle
\newpage
\tableofcontents
\newpage

%-----------------------------------prvni odstavec------------------------------------
\section{}
\paragraph{John von Neumannova a harvardská architektura počítače, princip jeho činnosti. Binární logika, logické operace a funkce, logické obvody. Reprezentace čísel a znaků v paměti počítače. Osobní počítač (PC), základní deska, chipset a sběrnice (interní, externí). Procesor (CPU), vykonávání instrukcí, podprogramy a zásobník, přerušení. Paměti počítače (RAM, cache, disk, diskové pole). Přídavné karty PC, datové mechaniky a média (CD, DVD, paměťové karty), periferie.}



\newpage
%-----------------------------------druhy odstavec------------------------------------
\section{}
\paragraph{Operační systém, architektura, poskytovaná rozhraní. Správa procesoru: procesy a vlákna, plánování jejich
běhu, komunikace a synchronizace. Problém uváznutí, jeho detekce a metody předcházení. Správa operační pa-
měti: segmentace, stránkování, virtuální paměť. Správa diskového prostoru: oddíly, souborové systémy, zajištění
konzistence dat.}




\newpage
%-----------------------------------treti odstavec------------------------------------
\section{}
\paragraph{Klasifikace (LAN/MAN/WAN) a služby počítačových sítí. Síťová architektura TCP/IP a referenční model
ISO OSI. Strukturovaná kabeláž, přepínaný Ethernet a WLAN. Protokol IP, adresa a maska, směrování, IP
multicast. Protokoly TCP a UDP, správa spojení a řízení toku dat. Systém DNS, překlad jména (na IP adresu,
reverzní), protokol. Služby WWW, elektronické pošty, přenosu souborů a vzdáleného přihlášení.}
\subsection{Klasifikace (LAN/MAN/WAN)}
Klasifikace sítí podle různých kritérií: rozlehlost, rychlost přenosu (klasické, vysokorychlostní), forma aplikace, dělení podle postavení uzlů (perr-to-peer, klient-server), podle druhu přenášení signálů (analog vs. digital) aj.
\subsubsection{Dělení podle rozlehlosti}
\begin{itemize}
	\item \textbf{PAN (Personal Area Network)}: Sítě s nejmenší rozlehlostí. Propojení mobilů, PDA, atd. Požadavky na odolnost vůči rušení, nízká spotřeba energie, snadná implementace.Ppřenosová rychlost není prioritou (zpravidla jen několik Mb/s). Dosah pouze několik metrů. Typické technologie Bluetooth, IrDA, Wifi.
	\item \textbf{LAN (Local Area Network)}: Sítě propojující koncové uzly typu počítač, tiskárna, server. V soukromé správě, dosah jednotky km (v rámci budovy/komplexu budov). Přenosové rychlosti 10Mb/s až 1Gb/s. Sdílené využití přenosového média. Ethernet, Wifi.
	\item \textbf{MAN (Metropolitan Area Network)}: Propojení několika LAN (účelem přenosové sítě, charakterem lokální). V rámci města (desítky km). Přenosové rychlosti jak vyšší (několik Gb/s), tak i nižší (<1Gb/s) ve srovnání s LAN.
	\item \textbf{WAN (Wide Area Network)}: Rozsáhlé sítě spojující LAN/MAN (páteřní sítě, telekomunikační - broadband). Velké vzdálenosti (prakticky neomezené). Mohou být soukromé i veřejné. Vysoké přenosové rychlosti (až stovky Gb/s). Příklad GPRS, xDSL, aj. Pronájem kapacity sítě = vyhrazené nesdílené využití přenosového média.
\end{itemize}
\subsubsection{Dělení podle topologie}
Topologie počítačové sítě říká, jak jsou vlastně prvky v této síti uspořádány.
\paragraph{Kruhová topologie (RING)}
Každý počítač je propojen přímo s předchozím a následujícím počítačem v kruhu. V LAN je používána velmi málo, používá se v průmyslových sítí nebo sítích MAN. \\ Výhody:
\begin{itemize}
	\item lehce rozšiřitelná struktura
	\item malý počet spojů
	\item snadné vysílání, zprávy v kruhu od stanice ke stanici
\end{itemize}
Nevýhody:
\begin{itemize}
	\item výpadek libovolné stanice zapříčiní výpadek celé sítě, úplný výpadek sítě při přerušení kabelu v  libovolném místě
	\item poměrně veliké nebezpeční odposlechu síťové komunikace, která prochází přes spojovací počítače
\end{itemize}
Problém výpadku se řešil tzv. dvojitým kruhem, ve kterém byly stanice propojeny dvěma kruhy, každý v opačné směru.
\paragraph{Sběrnicová topologie (BUS)}
Tato topologie patří k nejstarším, všechny stanice jsou připojeny na pasivní společné médium, které sdílejí. Dnes už se tato topologie příliš nepoužívá, ale na začátku devadesátých let byla dominantní. Tím společným médiem byl koaxiální kabel, pomocí kterého se jednotlivé počítače připojily do sítě. \\
Výhody:
\begin{itemize}
	\item nezávislost stanic na výpadku libovolné jiné stanice
	\item levné náklady takového řešení
	\item neexistence aktivních prvků
	\item snadné všesměrové vysílání
\end{itemize}
Nevýhody:
\begin{itemize}
	\item úplný výpadek sítě při přerušení kabelu v libovolném místě
	\item nutnost vyřešení přístupu stanic k médiu (kdo bude vysílat)
\end{itemize}
Výhody a nevýhody jsou relativní a poplatné době. To, že se v dobách používání této topologie v sítích LAN, považovala absence aktivních prvků za výhodu, bychom v dnešní době řadili spíše k nevýhodě.
\paragraph{Hvězdicová topologie}
Tato topologie je dnes jednoznačně nejpoužívanější topologií v sítích LAN. Myšlenka spočívá v tom, že existuje centrální prvek, který spojuje všechny prvky. Dříve tím centrálním prvkem býval počítač, dnes je aktivní prvek (HUB nebo SWITCH). \\
Výhody:
\begin{itemize}
	\item lehce rozšiřitelná struktura
	\item výpadek libovolné stanice neznamená výpadek celé sítě
	\item větší možnosti zabezpečení, při použití aktivních prvků typu SWITCH je většina síťové komunikace skryta před ostatními účastníky sítě
\end{itemize}
Nevýhody:
\begin{itemize}
	\item nutnost použití hubu nebo switche
	\item vyžaduje veliké množství kabelů a je tak náročná na montáž
\end{itemize}
\paragraph{Páteřní topologie}
Páteřní topologií rozumíme situaci, kdy pomocí určité topologie propojujeme celé sítě LAN. Páteřní topologie může být zapojena jako sběrnice, hvězda i kruh, často se používá zapojení typu kruh. Jejím základem je vytvoření nezávislé hlavní části, která propojuje důležité celky. Na ni se naopak připojují různé subsítě nebo segmenty. V případě výpadku libovolného segmentu zůstává provoz na páteři neohrožen. Páteř může mít vyšší přenosovou rychlost.
\subsubsection{Služby počítačových sítí}
\begin{itemize}
	\item připojení k síti
	\item vzdálený přístup, sdílení výpočetních prostředků a přenos dat (sdílené databáze, perr-to-peer sítě, sdílené soubory)
	\item sdílení technických prostředků (tiskárny, faxy, disky, apod.)
	\item adresářové služby (jednotný přístup do informačního systému a k informacím z centrální databáze, např. LDAP, Active Directory)
	\item elektronická pošta a výměna dokumentů (objednávky, faktury, atd.)
	\item online komunikace/multimedia (např. IRC, VoIP, straming, hry) - vysoké nároky na síť
	\item informační služby, internetové aplikace (WWW, business a desktopové aplikace)
	\item monitorování a vzdálená administrace sítě
\end{itemize}
Komunikace uzlů a propojovacích prvků sítě na různých úrovních:
\begin{itemize}
	\item nižší - přenos bloků dat, většinou nespolehlivý (bez potvrzení a opakování přenosu), nespojová komunikace
	\begin{itemize}
		\item \textbf{unicast}: dvoubodová, základní
		\item \textbf{multicast}: bod-skupina, např. straming, virtuální sítě
		\item \textbf{broadcast}: bod-všichni, např. konfigurace a zapojení do sítě
	\end{itemize}
	\item vyšší - komunikace aplikací, většinou spolehlivá (s potvrzením doručení, popř. opakování přenosu), spojově orientovaná (vytvořeno spojení mezi aplikacemi)
	\begin{itemize}
		\item \textbf{peer-to-peer}: zpravidla rovnocenná výměna dat
		\item \textbf{klient-server}: hiearchická, forma požadavek-odpověď
	\end{itemize}
\end{itemize}
Typy koncových uzlů:
\begin{itemize}
	\item \textbf{pracovní stanice} (work station, klient): převážně využívá služeb sítě
	\begin{itemize}
		\item \textbf{tenký klient}: znakový/grafický HW terminál, pouze zprostředkování vstupu a výstupu pro vzdálený server, nemůže pracovat samostatně
		\item \textbf{tlustý klient}: osobní počítač - klientské části síťových služeb i lokální úlohy, může (do určité míry) fungovat samostatně
	\end{itemize}
	\item \textbf{server}: převážně poskytuje služby v síti \\ souborový (FTP, NFS, SMB), databázový/adresářový (SŘBD, LDAP), poštovní (IMAP, POP3, SMTP), terminálový (telnet, SSH), informační/WWW (HTTP), komunikační (IM, VoIP), tiskový, aj.
\end{itemize}




\subsection{Síťová architektura TCP/IP a referenční model ISO OSI.}
Snaha o vytvoření univerzálního konceptu sítě (topologie, formy a pravidlakomunikace, poskytování služeb atd.). Požadavky: decentralizace služeb, rozumná adresace uzlů, data zasílána v nezávislých blocích, směrování bloků, zabezpečení, kontrola a řízení přenosu aj. Dříve proprietární uzavřená řešení, následně standatdizace s koncepcí komunikace nezávislé na implementaci. \\
\textbf{Komunikace ve vrstvách}
\begin{itemize}
	\item definované službami poskytované vyšším vrstvám a využívající služby nižších vrstev, implementace skryté okolním vrstvám
	\item samostatné vrstvy s funkcemi podobnými v rámci vrstvy a odlišnými v různých vrstvách, nezávislé na implementaci
\end{itemize}
Komunikace mezi vrstvami pomocí \textbf{mezivrstvových protokolů} - na každé komunikující straně zvlášť; skrze \textbf{programová rozhraní}; prostřednictvím přístupových bodů; využívající tzv. služební primitiva.
\paragraph{Obecná služební primitiva}
\begin{itemize}
	\item žádost o službu (request)
	\item oznámení poskytovatele o přijetí žádosti (indication) - nepovinné
	\item odezva poskytovatele (response), příp. vytvoření spojení
	\item potvrzení odezvy žadatelem (confirmation) - nepovinné
\end{itemize}
Komunikace ve stejných vrstvách mezi entitami (zařízeními) pomocí \textbf{vrstvových protokolů}.
\subsubsection{Protokol}
= souhrn pravidel (norem a doporučení) a procedur pro komunikaci (výměnu dat). Obsahuje syntaktická a sémantická pravidla výměny protokolových datových jednotek.\\
\textbf{Protokolové datové jednotky} = režijní informace a data (rámce, pakety, segmenty). Komunikace zprostředkována sousední nižší vrstvou. Na straně odesílatele \textbf{zapouzdřování} od nejvyšší po nejnižší vrstvu. Na straně příjemce \textbf{rozbalování} dat v opačném směru.
\paragraph{Síťová (protokolová) architektura} = definice vrstev, služeb funkcí, protokolů a forem komunikace. Normalizované (OSI, TCP/IP) a firemní proprietární (Novell NetWare, SMB, Apple Appletalk aj.)
\paragraph{Abstraktní referenční síťový model} Abstrakce konkrétních síťových architektur - nemusí podporovat všechny funkce modelu (např. průmyslové sítě nepodporují směrování, propojení pomocí mostů a bran).
\subsubsection{Referenční model ISO/OSI}
Propojení otevřených systémů = zařízení podporující příslušné normy. Deinovány koncové uzly (koncová datová zařízení) a mezilehlé uzly (propojovací prvky). \\
Každá ze sedmi vrstev vykonává skupinu jasně definovaných funkcí potřebných pro komunikaci. Pro svou činnost využívá služeb své sousední nižší vrstvy. Své služby pak poskytuje sousední vyšší vrstvě. Podle referenčního modelu není dovoleno vynechávat vrstvy, ale některá vrstva nemusí být aktivní. Takové vrstvě se říká nulová, nebo transparentní. \\
Na počátku vznikne požadavek některého procesu v aplikační vrstvě. Příslušný podsystém požádá o vytvoření spojení prezentační vrstvu. V rámci aplikační vrstvy je komunikace s protějším systémem řízena aplikačním protokolem. Podsystémy v prezentační vrstvě se dorozumívají prezentačním protokolem. Takto se postupuje stále níže až k fyzické vrstvě, kde se použije pro spojení přenosové prostředí. Současně se při přechodu z vyšší vrstvy k nižší přidávají k uživatelským (aplikačním) datům záhlaví jednotlivých vrstev. Tak dochází k postupnému zapouzdřování původní informace. U příjemce se postupně zpracovávají řídící informace jednotlivých vrstev a vykonávají jejich funkce.
\paragraph{Fyzická vrstva} Specifikuje fyzickou komunikaci. Aktivuje, udržuje a deaktivuje fyzické spoje mezi koncovými systémy. Definuje všechny elektrické a fyzikální vlastnosti zařízení (tvary konektorů; typy médií = kroucená dvojlinka, optické vlákno, mikrovlny; přenosové rychlosti). Obsahuje rozložení pinů, napěťové úrovně a specifikuje vlastnosti kabelů. \\
Funkce poskytované fyzickou vrstvou:
\begin{itemize}
	\item Navazování a ukončování spojení s komunikačním médiem.
	\item Spolupráce na efektivním rozložení všech zdrojů mezi všechny uživatele.
	\item Modulace neboli konverze digitálních dat na signály používané přenosovým médiem (a zpět) (A/D, D/A převodníky).
\end{itemize}
HW zařízení: HUB, opakovač.
\paragraph{Linková vrstva} Poskytuje spojení mezi dvěma sousedními systémy. Uspořádává data z fyzické vrstvy do logických celků = \textbf{rámce (frames)}: záhlaví s linkovou adresou příjemce a odesílatele (např. MAC) + data + zápatí s kontrolním součtem (CRC) celého rámce. Stará se o nastavení parametrů přenosu linky, oznamuje neopravitelné chyby. Formátuje fyzické rámce, opatřuje je fyzickou adresou a poskytuje synchronizaci pro fyzickou vrstvu. Příkladem je MAC u Ethernetu. Poskytuje propojení pouze mezi místně připojenými zařízeními a tak vytváří doménu na druhé vrstvě pro směrové a všesměrové vysílání. \\
HW zařízení: most, přepínač, síťová karta, přístupový bod.
\paragraph{Síťová vrstva} Stará se o směrování v síti a síťové adresování. Poskytuje spojení mezi systémy, které spolu přímo nesousedí. Poskytuje funkce k zajištění přenosu dat různé délky od zdroje k příjemci skrze jednu případně několik vzájemně propojených sítí při zachování kvality služby, kterou požaduje přenosová vrstva. Síťová vrstva poskytuje směrovací funkce a také reportuje o problémech při doručování dat. Jednotkou přenosu je \textbf{síťový packet}: záhlaví se síťovou adresou příjemce a odesílatele (např. IP) + data + zápatí jen výjimečně. \\
Funkce poskytované síťovou vrstvou:
\begin{itemize}
	\item abstrakce různých linkových technologií
	\item správa linkových spojení, multiplexování síťových spojení do linkových (více datových toků kombinováno do jednoho)
	\item formátování dat do packetů
	\item směrování packetů
	\item zjišťování a oprava chyb
	\item vytváření podsítí
\end{itemize}
HW zařízení: směrovač, brána.
\paragraph{Transportní vrstva} Zajišťuje přenos dat mezi koncovými uzly. Jejím účelem je poskytnout takovou kvalitu přenosu, jakou požadují vyšší vrstvy. Vrstva nabízí spojově (TCP) a nespojově orientované (UDP) protokoly. Jednotkou přenosu je \textbf{transportní packet}: záhlaví s transportní adresou příjemce a odesílatele + data. \\
\textbf{TCP}: Protokol garantuje spolehlivé doručování a doručování ve správném pořadí. TCP také umožňuje rozlišovat a rozdělovat data pro více aplikací (například webový server a emailový server) běžících na stejném počítači. TCP využívá mnoho populárních aplikačních protokolů a aplikací na internetu, včetně WWW, e-mailu a SSH. \\
\textbf{UDP}: Na rozdíl od protokolu TCP nezaručuje, zda se přenášený datagram neztratí, zda se nezmění pořadí doručených datagramů, nebo zda některý datagram nebude doručen vícekrát. UDP je vhodný pro nasazení, které vyžaduje jednoduchost nebo pro aplikace pracující systémem otázka-odpověď (např. DNS, sdílení souborů v LAN). \\
Funkce poskytované transportní vrstvou:
\begin{itemize}
	\item adresování (transportní na síťové)
	\item správa síťových spojení
	\item multiplexování a větvení
	\item rozdělení dat na datagramy, segmentace, formátování
	\item řízení proudu dat (správné pořadí datagramů), optimalizace služeb
	\item koncová detekce a oprava chyb
\end{itemize}
Umožňuje \textbf{duplexní přenos} (= přenos oběma směry).
\paragraph{Relační vrstva} Smyslem vrstvy je organizovat a synchronizovat dialog mezi spolupracujícími relačními vrstvami obou systémů a řídit výměnu dat mezi nimi (např. sdílení síťového disku). Umožňuje vytvoření a ukončení relačního spojení, synchronizaci a obnovení spojení, oznamovaní výjimečných stavů. Jednotka přenosu je \textbf{relační packet}: poze data. \\
Funkce poskytované relační vrstvou:
\begin{itemize}
	\item organizace a synchronizcae dialogu výměny dat (pomocí kontrolních bodů)
	\item zobrazení relačních spojení do transportních
	\item správa transportních spojení
\end{itemize}
\paragraph{Prezentační vrstva} Funkcí vrstvy je transformovat data do tvaru, který používají aplikace (šifrování, konvertování, komprimace). Formát dat (datové struktury) se může lišit na obou komunikujících systémech, navíc dochází k transformaci pro účel přenosu dat nižšími vrstvami. Vrstva se zabývá jen strukturou dat, ale ne jejich významem, který je znám jen vrstvě aplikační. \\
Funkce poskytované prezentační vrstvou:
\begin{itemize}
	\item transformace a výběr reprezentace dat
	\item formátování, komprese, zabezpečení (šifrování), integrita dat
	\item transparentní přenos zpráv (nezná jejich význam
\end{itemize}
\paragraph{Aplikační vrstva} Účelem vrstvy je poskytnout aplikacím přístup ke komunikačnímu systému a umožnit tak jejich spolupráci. \\
Funkce poskytované aplikační vrstvou:
\begin{itemize}
	\item zprostředkování funkcionality sítě
	\item přenos zpráv, určení kvality, synchronizace
	\item identifikace, stanovení pověření
	\item dohoda o ochraně, o opravách chyb
\end{itemize}
Protokoly: SMTP, SSH, Telnet, POP3, DHCP, FTP, DNS aj.
\paragraph{Funkce společné více  vrstvám}
Komunikace \textbf{se spojením} má 3 fáze: navázání spojení, přenos dat, ukončení spojení. Dohoda na parametrech, použití potvrzování přijetí/nepříjetí (spolehlivost), stejné pořadí dat na vstupu i výstupu. \\
Komunikace \textbf{bez spojení}: při každém přenosu všechny parametry, nezávislý přenos datových jednotek, různé pořadí na vstupu a výstupu, spolehlivost i nespolehlivost. \\
Dále se může mezi těmito typy konvertovat. Transportní služby musí být se spojením.
\subsubsection{TCP/IP}
použití v síti Internet, nejpoužívanější síťová architektura. Síť tvořena směrovači, specializovanými bránami (bezpečnostní, aplikační, telekomunikační), lokálními sítěmi a koncovými zařízeními. Vlastní protokoly.
\paragraph{Vrstva síťového rozhraní} Nejnižší vrstva umožňuje přístup k fyzickému přenosovému médiu. Je specifická pro každou síť v závislosti na její implementaci.
\paragraph{Síťová vrstva} Vrstva zajišťuje především síťovou adresaci, směrování a předávání datagramů.
\paragraph{Transportní vrstva} je implementována až v koncových zařízeních (počítačích) a umožňuje proto přizpůsobit chování sítě potřebám aplikace. Poskytuje transportní služby kontrolovaným spojením spolehlivým protokolem TCP nebo nekontrolovaným spojením nespolehlivým protokolem UDP.
\paragraph{Aplikační vrstva} Vrstva aplikací. To jsou programy (procesy), které využívají přenosu dat po síti ke konkrétním službám pro uživatele. \\
Aplikační protokoly používají vždy jednu ze dvou základních služeb transportní vrstvy: TCP nebo UDP, případně obě dvě (např. DNS). Pro rozlišení aplikačních protokolů se používají tzv. porty, což jsou domluvená číselná označení aplikací. Každé síťové spojení aplikace je jednoznačně určeno číslem portu a transportním protokolem (a samozřejmě adresou počítače).


\end{document}

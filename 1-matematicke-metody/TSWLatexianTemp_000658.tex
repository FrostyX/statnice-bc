\documentclass[12pt,a4paper]{article}
\usepackage[utf8]{inputenc}
\usepackage[czech]{babel}
\usepackage[T1]{fontenc}
\setlength{\parskip}{12pt}
\usepackage{ amssymb }
\usepackage{ mathrsfs }
\usepackage{amsthm}
\usepackage{amsmath}
\newtheorem{definition}{Definice}
\newtheorem{example}{Příklad}

\begin{document}

\title{Státnicový okruh 1: \\ Matematické metody}
\maketitle
\newpage
\tableofcontents
\newpage

\section{Množiny}
Množina je objekt, který se skládá z jiných objektů tzv. \textbf{prvků} té množiny. Množiny zpravidla značíme velkými písmeny (A, B, \dots, Z), jejich prvky pak malými písmeny (a, b, \dots, z). Fakt, že $x$ je prvkem množiny $A$ značíme $x \in A$. Není-li prvkem $A$ značíme $x \not\in A$.

Množina je jednoznačné dána svými prvky. Prvek do množiny buď patní nebo ne. Nemá tedy smysl hovořit o pořadí prvků a také nemá smysl zabývat se tím kolikrát se daný prvek v množině nachází.
Speciální množinou je tzv. \textbf{prázdná množina} značíme $\emptyset$. Tato množina neobsahuje žádné prvky tedy pro všechna $x$ platí, že $x \not\in \emptyset$.

\subsection{Dělení množin}
Množiny dělíme na \textbf{konečné} a \textbf{nekonečné}. Množina $A$ se nazývá konečná právě když existuje přirozené číslo $n$ tak že prvky této množiny můžeme očíslovat čísly 1, 2, \dots, $n$. Číslo $n$ nazveme počet prvků množiny a značíme jej $|A|$
Pokud $|A| = \infty$ nazveme množinu nekonečnou a říkáme, že má nekonečně mnoho prvků.

\subsection{Zapisování množin}
Množiny můžeme zapisovat následujícími způsoby:
\begin{enumerate}
	\item \textbf{Výčtem prvků} - Množinu která obsahuje prvky $a_1,a_2,\dots,a_n$ zapíšeme následovně $\{a_1,a_2,\dots,a_n\}$.
	\item \textbf{Pomocí charakteristické vlastnosti} - Množina obsahuje právě ty prvky, které splňují vlastnost $\varphi(x)$ zapisujeme $\{x | \varphi(x)\}$. Vlastnost $\varphi(x)$ může být popsána i slovně. Příklad $\varphi(x)$: číslo x je sudé.
\end{enumerate}
\newpage
\subsection{Vztahy mezi množinami}
Základními vztahy mezi množinami jsou \textbf{rovnost} (=) a \textbf{inkluze} ($\subseteq$)
\begin{itemize}
	\item[] $A = B$ znamená, že pro každé $x : x \in A$ právě když $x \in B$
	\item[] $A \subseteq B$ znamená, že pro každé $x$ : jestliže $x \in A$ pak $x \in B$
	\item[] $A \not= B$ znamená že neplatí $A = B$
	\item[] $A \not\subseteq B$ znamená, že neplatí $A \subseteq B$
\end{itemize}

Množina jejichž prvky jsou právě všechny podmnožiny dané množiny $X$, nazýváme \textbf{potenční množina} množiny $X$ značí se $\mathscr{P}(X)$ nebo také $2^X$. Tedy $2^X = \{A|A\subseteq X\}$.

\subsection{Operace s množinami}
Mezi základní operace s množinami patří průnik (značí se $\cap$), sjednocení (značí se $\cup$), rozdíl (značí se $\setminus$).

Jsou-li $A,B$ množiny, definujeme množiny $A \cup B, A \cap B, A \setminus B$ následovně:
\begin{itemize}
	\item[] $A \cap B = \{ x|x \in A \text{ a } x \in B\}$ 
	\item[] $A \cup B = \{ x|x \in A \text{ nebo } x \in B\}$ 
	\item[] $A \setminus B = \{ x|x \in A \text{ a } x \not\in B\}$ 
\end{itemize}

Množiny nazýváme navzájem disjunktní právě když $A \cap B = \emptyset$.

\subsubsection{Vlastnosti operací}
\begin{itemize}
	\item $A \cap \emptyset = \emptyset, \hspace{10pt} A \cup \emptyset = A,\hspace{10pt} A \cap A = A$
	\item $A \cup B = B \cup A, \hspace{10pt} A \cap B = B \cap A$
	\item $(A \cup B) \cup C = A \cup (B \cup C)$
	\item $A \cap (B \cup C) = (A \cap B) \cup (A \cap C), \hspace{10pt} A \cup (B \cap C) = (A \cup B) \cap (A \cup C)$ 
	\item $A \cup (A \cap B) = A, \hspace{10pt} A \cap (A \cup B) = A$ 
\end{itemize}

\section{Relace}
Pojem relace je matematickým protějškem pojmu \textit{vztah}. Různé objekty jsou nebo nejsou v různých vztazích. Například číslo 3 je ve vztahu \uv{být menší} s číslem 5. Vztah je také určen aritou tj. počtem objektů které do vztahu vstupují, výše uvedený příklad má aritu 2 protože porovnáváme 2 čísla. Takovou relaci nazveme binární. Dále máme unární (jeden prvek), ternární (tři prvky), \dots

\begin{definition}
Kartézský součin množin $X_1, X_2, \dots,X_n$ je množina $X_1 \times X_2 \times \dots \times X_n$ definovaná předpisem $$X_1 \times \dots \times X_n =  \{\langle x_1, \dots x_n \rangle | x_1 \in X_1, \dots, x_n \in X_n\} $$
\end{definition}
 Kartézský součin $n$ množin je množina všech uspořádaných n-tic prvků z těchto množin. Je-li $X_1 = \dots = X_n = X$ pak $X_1 \times \dots \times X_n$ značíme také $X^n$ (n-tá kartézská mocnina množiny $X$)

\begin{definition}
Nechť $X_1, \dots,X_n$ jsou množiny. Relace mezi $X_1, \dots,X_n$ je libovolná podmnožina kartézského součinu $X_1 \times \dots \times X_n $.
\end{definition}

\begin{example}
Mějme množinu $A = \{1,2,3,4\}$ a množinu $B = \{a,b,c,d\}$. Relace $R,S \subseteq A \times B$ mohou vypadat následovně. $$R = \{\langle 1, a\rangle, \langle 1, b\rangle, \langle 3, d\rangle, \langle 4, a\rangle\}$$
$$S = \{\langle 3, a\rangle, \langle 1, c\rangle\}$$
\end{example}


\end{document}
